%Este comando hace que la página que contiene la introducción sea la número uno
\setcounter{page}{1}
\section*{Introducción}%introducción

Las redes computacionales son un conjunto 
de distintos equipos conectados entre sí mediante diferentes medios ya sea de manera alámbrica o innalámbrica,
 de tal forma que les permite cumplir con el propósito de compartir información con la cuál se es posible abarcar 
 distintos aspectos, como lo pueden ser la búsqueda de información para investigación, recolectar y almacenar datos personales, 
 obtener servicios tales como el entretenimiento u otras apliciones empresariales. Hoy en día las redes abarcan un aspecto tan 
 importante en la vida cotidiana que ha sido de gran importancia ser capaz de administrarla de una manera ágil y eficiente para 
 poder satisfacer a todos aquellos usuarios que requieran hacer uso de ella y obtener una buena experiencia, 
 debido a esta gran importancia es que se han creado distintas entidades con las cuáles ser capaz de lograrlo, 
 dicha entidad es conocida como el administrador de la red. Con el paso del tiempo y el surgimiento de distintas tecnologías,
  las redes computacionales han tenido que también evolucionar para solventar las nuevas necesidades que surgen.



La importancia de la facilidad y rapidez en la administración de la red por parte del administrador recae en que no existe una 
red que sea perfecta, e inevitablemente surgen problemas o nuevas configuraciones que deben de poder ser resueltas en el menor 
tiempo posible lo cuál podría lograrse haciendo que esto sea cada vez más fácil. Con el paso del tiempo, se ha buscado lograr la
 automatización de la red, esto mediante la creación de una tecnología que sea capaz de detectar y solucionar problemas en la red,
así como también otorgar al administrador
herramientas que le permitan realizar e implementar las configuraciones deseadas. Con la finalidad de lograr estas funciones es 
que se han desarrollado controladores de red, herramientas de software capaces de comunicarse con los distintos dispositivos de red
enviando instrucciones y recibiendo información de tal forma que en conjunto se puede administrar la red.

La creación de los controladores de red permitió la idea de \textbf{redes definidas por software (SDN)}, las cuales emplean 
herramientras de software como aplicaciones y dispositivos especializados para 
ser capaces de implementar servicios y desplegar aplicaciones de la red, permitiendo así el avance en el desarrollo en los
distintos aspectos que son de gran interés en toda red, tales como la escalabilidad, seguridad, administración, entre otros.
Las SDN proporcionan al administrador de la red una manera inteligente, fácil y más rápida de supervisar, operar y realizar
las modificaciones que el momento así lo demande en la red dada las necesidades que ocurran. 

Las \textbf{SDN} nos brindan una gran y amplia gama de servicios y beneficios en una red. Con ellas se es capaz de
simplificar la administración de la red, implementar redes complejas en un tiempo considerablemente menor al que tomaría
hacerlo de otra manera, disminuir los costos de mantenimiento mediante la disminución del tiempo necesario para ser 
administradas e implementadas, disminuir el uso de hardware específico. Otorgan la habilidad de programar configuraciones
automáticas incrementando así la escalabilidad y rentabilidad, así como también solucionar problemas al instante, 
reducir los costos y ayudar a innovar en nuevas aplicaciones y servicios para la red.  

Las \textbf{SDN}  han permitido que cada vez más personas sean capaces de llevar a cabo las actividades
que son requeridas en una red, puesto que el software es aquel que se se encarga de llevar a cabo las tareas, disminuyendo así la 
posibilidad de un error humano. 
Es así entonces, que se desarrollará una aplicación web que utilice las librerías de código abierto en Python llamada Netmiko y 
Paramiko, desde la cuál un administrador de
red será capaz de desplegar servicios en ella de tal manera que no sea necesario que tenga que conectarse 
físicamente a los dispositivos, si no que mediante la ténica de "drag and drop" será capaz de implementarlos en los equipos que se 
encuentren en la red a
administrar de manera remota.


% \textit{Wavelet}  texto raro



\section*{Justificación}

Las redes de computadoras se encuentran implementadas y operando en distintos ámbitos, realizando importantes tareas
para muchos sectores de la sociedad, organizaciones públicas o privadas son sólo algunos de los 
lugares en donde las redes computacionales son vitales para el correcto funcionamiento de sus actividades. Mediante el uso de las SDN las redes tendrían un 
notable mejoramiento en distintas actividades que le conciernen a ésta, tales
como su administración, implementación y mantenimiento. Es así como las SDN mejoran el control de una red, permitiendo separar el 
plano de control del plano de datos. A su vez, estas permitirían una independencia del tipo del hardware y del proveedor de equipos, pues todo pasaría a ser 
responsabilidad de el software implementado en cuestión. El nuevo paradigma de las SDN es relativamente nuevo, por lo que aún se 
encuentra en desarrollo el cómo debe estar constituido, pero todo apunta a que las SDN son el futuro de las redes.

Llevar a cabo una aplicación que sea capaz de implementar servicios en una red tendría la importancia de investigar y aportar 
a una nueva tecnología vanguardista. Una aplicación capaz de 
desplegar servicios desde un punto de vista de alto nivel facilitaría la tarea para el administrador, puesto que de esta
forma se ahorraría tiempo que le toma a un administrador realizar cada una de las configuraciones,así 
como se evitarían posibles errores que este podría cometer, como lo son errores en la sintaxis. Hacer que dicha 
aplicación opere en una plataforma web ahorraría la necesidad de encontrarse físicamente en el lugar de la red a administrar,
pudiendo así manipularla desde cualquier parte. Al mismo tiempo tiene una importancia al momento de 
capacitar personal para adminstrar una red, pues ya no sería necesaria la enseñanza de comandos técnicos como lo es el uso de la CLI 
de cada uno de los equipos.

Debido a lo antes mencionado, implementar una aplicación  web que sea capaz de administrar una red, implementar los servicios en 
ella, así como también darle un debido mantenimiento es de gran importancia para satisfacer dichas necesidades.
\section*{Objetivos}

\subsection*{Objetivo General}
Diseño e implementación de una aplicación web para la administración de servicios en redes definidas por software.

\subsection*{Objetivos Específicos}
\begin{enumerate}
  \item Diseñar e implementar un módulo para comunicar la aplicación con el controlador de la red.
  \item Diseñar e implementar un módulo que recupere los datos necesarios sobre la topología de la red.
  \item Diseñar e implementar un módulo para el despliegue en pantalla de la topología de la red.
  \item Diseñar e implementar un módulo para detectar cambios en la topología de la red.
  \item Diseñar e implementar un módulo para traducir instrucciones del usuario a comandos para los dispositivos de red.
  \item Diseñar e implementar un módulo que envíe las instrucciones a los distintos dispositivos de red.
  \item Diseñar e implementar los algoritmos para el despliegue de al menos 3 servicios en la red. 
\end{enumerate}

\section*{Trabajos Relacionados}

La adminstración de las redes abarca muchos temas de interés ya que se encuentran en muchas distintas áreas. 
Ya sea al nivel del software como del hardware las aplicaciones utilizadas para la administración de las redes proporciona
 avances de distinta índole. Alguno trabajos que se encuentran relacionados con este tema son los siguientes:




\begin{itemize}
    \item \textbf{``Diseño e implementación de una aplicación
    para la Administración de una Red en 1.0 Linux''  \cite{Rodriguez10}.} 
    Aplicación computacional diseñada para  la gestión de la seguridad de una red mediante un router
    de tecnología Cisco.    
    \item \textbf{``Diseño e Implementación de una Aplicación para
    la Administración del una Red en el SO Linux''  \cite{Garduno11}.}
     Aplicación para la administración de una red en el Sistema Operativo Linux mediante el uso de SNMP (Simple Network Managment Protocol)
     \item \textbf{''Sistema para la visualización gráfica de la topología de una red de computadoras''\cite{Alcantara15}.}
     Aplicación de escritorio para la visualización de una red área local mediante el protocolo SNMP (Simple Network Managment Protocol) y el uso del lenguaje de programción Java.   
    \item \textbf{``Diseño y desarrollo de una aplicación gráfica para la
    configuración de redes virtuales de área local en un sistema
    operativo LINUX" \cite{Marquez2013}.}
    El objetivo principal es crear una nueva interfaz gráfica para la creación de redes virtuales de área local utilizando herramientas que sólo pueden ser utilizadas en la termianl de sistemas Linux.
    \item \textbf{``HERRAMIENTAS DE SOFTWARE LIBRE PARA EL
    MONITOREO DE ACTIVIDADES DE USUARIOS EN
    REDES LAN'' \cite{Dino2018}.}
    Realiza un informe detallado sobre la comparación de herramientras de software libre para monitorear actividades que ocurren en la red, esto mediante el uso e interpretación de distintos protocolos 
    que son implementados para la seguridad en las redes.

\end{itemize}


\newpage
\begin{table}[ht!]
  \begin{tabular}{p{0.15\textwidth} p{0.4\textwidth} p{0.4\textwidth}}
    \toprule
    \textbf{{Referencia}} & \textbf{{Similitudes}} & \textbf{{Diferencias}} \\
    \toprule
    \cite{Rodriguez10} &
    \begin{itemize}[leftmargin=*]
        \item El sistema es desarrollado en un ambiente Linux.
        \item El objetivo del sistema implementado.
    \end{itemize} &
    \begin{itemize}[leftmargin=*]
        \item Uso de un protocolo distinto.
        %\item Distinto enfoque tem\'atico.
    \end{itemize} \\
    \midrule
    \cite{Garduno11} &
    \begin{itemize}[leftmargin=*]
        \item Implementación de un servidor.
        \item Implementación de tecnologías web.
    \end{itemize} &
    \begin{itemize}[leftmargin=*]
        \item Uso del protocolo SNMP. 
        \item Ausencia de despliegue de la topolía de la red.
    \end{itemize} \\
    \midrule
    \cite{Alcantara15} &
    \begin{itemize}[leftmargin=*]
        \item Visualización de la topolía de la red.
    \end{itemize} &
    \begin{itemize}[leftmargin=*]
        \item Uso de SNMP.
        \item Desarrollado con lenguaje de programación Java.
    \end{itemize} \\
     \midrule
    \cite{Marquez2013} &
    \begin{itemize}[leftmargin=*]
        \item Tratamiento de se\~nales EEG.
        \item Identificaci\'on de movimientos corporales.
    \end{itemize} &
    \begin{itemize}[leftmargin=*] 
        \item Ausencia de implementaci\'on.
    \end{itemize} \\
    \midrule
    \cite{Dino2018} &
    \begin{itemize}[leftmargin=*]
        \item An\'alisis de componentes de se\~nales EEG.
    \end{itemize} &
    \begin{itemize}[leftmargin=*]
        \item Ausencia de implementaci\'on.
    \end{itemize} \\
    \midrule
        \cite{Chacon14} &
    \begin{itemize}[leftmargin=*]
        \item Procesamiento de se\~nales.
        \item Implementaci\'on en FPGA.
    \end{itemize} &
    \begin{itemize}[leftmargin=*]
        \item Tipo de se\~nales a procesar. 
    \end{itemize} \\
    \bottomrule
  \end{tabular}
  \caption{Comparación cualitativa de los trabajos relacionados con el proyecto propuesto.}
  \label{table:related}
\end{table}
\newpage

%Cerrar esta sección con una conclusión
%\begin{center}
 % \begin{minipage}{0.4\textwidth}
  %  \centering
   % M. en C. Oscar Alvarado Nava %grado y nombre completo del asesor
  %\end{minipage}
   % \begin{minipage}{0.4\textwidth}
    %\centering
   % Dr. Eduardo Rodr\'iguez Mart\'inez%grado y Nombre completo del coasesor
  %\end{minipage}
  %\newpage
%\end{center}
